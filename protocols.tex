%        File: durps.tex
%     Created: Thu Jun 26 10:00 PM 2014 S
% Last Change: Thu Jun 26 10:00 PM 2014 S
%
\documentclass[a4paper]{article}
%\documentclass{scrartcl}
%\usepackage[utf8]{inputenc}
%\usepackage[english]{babel}
\usepackage{textcomp}
\usepackage{titlesec}
\usepackage{multicol}
\usepackage[cm]{fullpage}
\usepackage[]{graphicx}
\usepackage{booktabs}
\usepackage{tabularx}
%\setlength{\parindent}{0pt}
\usepackage{txfonts}
%\usepackage{fancyhdr}
%\pagestyle{fancy}
%\renewcommand{\headrulewidth}{0pt}
%\cfoot{\textcopyright 2014 Johann Eicher}

\newcommand{\dc}{~$^{\circ}$C}
\newcommand{\ec}{\textit{E. coli}}
\newcommand{\micro}{$\muup$}
\newcommand{\od}{OD$_{600}$}

\begin{document} \begin{center} \huge Introductory tutorial on bacterial growth
and kinetic assays in the Molecular Systems Biology Group\\ \normalsize
\end{center} 

\begin{multicols}{2} This tutorial serves as a practical introduction to the
typical experimental work performed in the Molucalar Systems Biology lab. \ec\
cells will be grown from freezer stocks, and cell lysates will be prepared and
used in a simple lactate dehydrogenase kinetic assay.

It is assumed that the person performing these experiments will have received
proper instruction with regard to the safety protocols and equipment booking
systems in the department.

\section{Sterile technique} 

When working with microorganisms almost all work is done under sterile
conditions to prevent contamination, and solutions prepared under non-sterile
conditions are subsequently sterilised by autoclaving. Contaminating bacteria,
yeast and fungi live on almost everything and float around in the laboratory
air. When using sterile technique, the following minimal guidelines apply:

\begin{enumerate} 
\item Work in the laminar flow hood which provides positive
pressure (make sure the sterilising UV light is turned off whilst working the
hood, and on otherwise).  
\item Always wear latex gloves OR regularly sterilise
hands with a 70~\% ethanol solution.  
\item Flame bottlenecks with a bunsen
burner whenever they are opened or used, and flame inoculating loops before
each inoculation. Hockey-sticks should be kept in 70~\% ethanol and flamed
before use.  
\item Only use autoclaved pipette tips.  
\end{enumerate}

Note also that when using shakers and centrifuges to observe correct protocol
and ensure that flasks or centrifuge rotors are balanced.

\section{Growing \textit{Escherichia coli}} 

\ec\ is typically grown in LB medium for bulk preparation of cell matter,
however for reproducible physiological and/or kinetic studies a defined medium
such as M9 minimal medium has to be used.

To demonstrate the typical growth process of \ec\ the following steps will be
taken:

\begin{enumerate}
\item prepare either LB or M9 medium
\item inoculate a 5~ml culture from a freezer stock of \ec\ and allow it to grow overnight (O/N)
\item use the O/N culture to inoculate a large culture and incubate on a shaker at 37\dc
\item monitor cell growth using a spectrophotometer
\end{enumerate}

The following general guidelines for preparing LB or M9 medium should be
adjusted to prepare the desired amount of medium. For the purposes of this
introduction a 50~ml overnight culture and a large 500~ml culture will be
sufficient.

The general buffer for cell washing, lysis, and kinetic assays used in this
introduction is 100~mM TRIS at pH~7.0. 500~ml of this buffer is sufficient and
should be prepared beforehand and autoclaved (121\dc\ for 15-20 min).

\subsection{Lysogeny broth (LB) medium} 

This is a rich medium composed of tryptic digests of casein (peptides), yeast
extract (vitamins, trace elements), and NaCl. As the exact composition of
constituents in tryptone and yeast extracts is unknown and will vary per batch,
LB medium is not recommended for physiological and kinetic studies, but rather
for bulk growing of bacteria (e.g. for protein expression).

To prepare 1~l of LB medium, dissolve the following components in water
(distilled water or preferably milli-Q) in a beaker with a magnetic spinner:

\begin{center}
\begin{tabular}[h]{rl} \toprule
10~g    &tryptone \\ 
5~g     &yeast extract\\ 
10~g    &NaCl\\ \bottomrule
\end{tabular}
\end{center}

As a precaution it is possible at this stage to adjust the pH of the solution
but is probably unnecessary. The solution can now be decanted into Erlenmeyer
flasks, each with a cotton wool bung and foil covering the mouth (note that for
adequate aeration a 1~l culture should be grown in a 3~l flask, and
a 500~ml culture in a 2~l flask) and autoclaved (usually at 121\dc\ for 15-20
min).

\subsection{M9 minimal medium}

M9 is a minimal medium with clearly defined components making it ideal for
reproducible physiological and kinetic studies. Minimal media force the
organism to produce the majority of the required metabolites, vitamins,
and co-factors endogenously and only provides the bare minimum of components
exogenously in the medium. Unlike LB medium, M9 medium is buffered by the two
phosphate components HPO$_4^{2-}$ and H$_2$PO$_4^{1-}$. It is thus imperative
that care is taken during preparation to include the correct forms of the
phosphate salt!

To prepare 1~l of M9 medium, dissolve the following in 977~ml of water (milli-Q is
preferred for reproducibility) in a beaker with a magnetic spinner:

\begin{center}
\begin{tabular}[h]{rl} \toprule
12.8 g  & Na$_2$HPO$_4$.7H$_2$O \textbf{OR}\\
6.0 g  & Na$_2$HPO$_4$\\\midrule
3.0 g  & KH$_2$PO$_4$\\
0.5 g   & NaCl\\
1.0 g   & NH$_4$Cl\\ \bottomrule
\end{tabular}
\end{center}

Also prepare a 20\% w/v glucose solution in milli-Q water (typically glucose,
sodium gluconate or glycerol are used as carbon sources); 500~ml is enough for
25~l of M9 medium.  Glucose solutions may take a while to dissolve fully, but
if left stirring with a magnetic stirrer the clumps will eventually disperse.

Transfer the M9 salts and glucose solutions to Schott bottles and prepare some
Erlenmeyer flasks for culturing by inserting a cotton wool bung into the mouth
of each flask and cover the mouth with foil (note that for adequate aeration a
1~l culture should be grown in a 3~l flask, and a 500~ml culture in a 2~l
flask). Autoclave these solutions and empty flasks (usually at 121\dc\ for
15-20 min).  The reason for autoclaving the salts and sugars separately is that
when autoclaved together the sugars are often caramelised, especially when
phosphates are present.

While the glucose and salt solutions are being autoclaved, prepare the
following two constituents. Once prepared, they should be filter sterilised
using a 50~ml syringe and a 0.2~\micro m syringe filter into 50~ml
Falcon tubes and stored at 4\dc\ (use a new filter for each solution, the
solutions can be reused for subsequent cultures as long as they do not become
contaminated):

\begin{itemize}
\item 50~ml 1~M MgSO$_4$
\item 50~ml 0.1~M CaCl$_2$
\end{itemize}

To prepare 1~l of M9 medium add the following under sterile conditions to the
977~ml M9 salts autoclaved salts in the Schott bottle:

\begin{center}
\begin{tabular}[h]{rl} \toprule
20~ml   & 20~\% carbon source (0.2~\% final)\\
2~ml    & 1~M MgSO$_4$\\
1~ml    & 0.1~M CaCl$_2$\\ \bottomrule
\end{tabular}
\end{center}

As a precaution it is possible at this stage to adjust the pH of the solution
which will be buffered around 7.2 by adding sterilised NaOH/HCl. This is
generally unncecessary for growth of \ec.

This M9 medium may now be decanted into Erlenmeyer flasks for culturing.

\subsection{Growing cells on agar plates (optional)} 

Often it is useful to grow bacteria on agar plates.
We do this regularly to check for contamination, do cell counts, or simply to
have an intermediate source of cells so that we don't have to constantly use
freezer stocks.

To prepare plates, either LB or M9 medium is prepared as above with the
exception that 15~g of agar are added per litre of medium. This new medium is
autoclaved in a Schott bottle and once it has cooled enough to be handled (but
still liquid), the medium is poured into petri dishes in the laminar flow hood
under sterile conditions. Petri dishes should only be filled halfway and left
slight uncovered in the laminar flow hood (ideally with the UV light on) will
setting to prevent condensation on the lid. These plates can be stored at
4~\dc\ for several weeks by wrapping them in parafilm to prevent dehydration.

There are several ways to grow bacteria on plates. Typically, bacterial
cultures are of far too high an \od\ to apply directly to a plate as the
density of cells will prevent individual colonies from being identified. Thus
it is usually necessary to perform a dilution series (e.g. in pre-sterilised
Eppendorf tubes) until individual colonies on the plate are identifiable.

If a simple check for contamination is required, an inoculation loop can be
used to inoculate a plate by flaming it, dipping it into a diluted cell
suspension, and streaking out several lines in one quadrant of the plate. After
flaming the loop again, the cells can be spread out further by streaking out
several new lines through the old lines. This should be done four times (once
for each quadrant of the plate).

Alternatively, if a cell count is required, the same dilution series of a
culture is made and 200~ul of each dilution is pipetted directly onto a plate.
After this, a glass ``hockey stick'' is removed from the 70~\% ethanol in which
it is kept, and flamed. Once the flames have extinguished, the flat frontal
part of the ``hockey stick'' is used to smear the 200~ul of cells evenly over
the surface of the plate. Once colonies have appeared, they are counted and
multiplied by the appropriate dilution factor to estimate the number of cells
in the original culture (1 colony forming unit or CFU represents 1 cell in the
original culture).

Agar plates are incubated at 37\dc upside down on a shelf (to prevent
condensation dripping onto the cells).


\subsection{Batch growth -- overnight cultures} 

Bacterial stocks are typically frozen at -80\dc\ in a 50~\% glycerol solution
to protect the cells from being damaged by ice crystals. To bring the cells out
of the deep stationary phase they experience at such low temperatures we
usually start by growing a small O/N culture which is used on the following day
to inoculate a larger culture. Inoculating directly from freezer stocks is
possible, but typically involves such a long lag period as the cells ``wake
up'' that it is more convenient to start with O/N culturing. Additionally, O/N
culturing allows one to predefine the starting optical density of a culture.

Typical aerobic bacterial batch cultures are grown by
shaking in Erlenmeyer flasks. \ec\ are grown in a 37\dc\ room with sufficient
shaking to continuously ``fold'' air into the medium.

To inoculate a 50~ml O/N culture, fetch a freezer stock of \ec\ cells in an
ice-box. After allowing the sample to thaw, using sterile technique in the
laminar flow hood, inoculate the 50~ml culture with 20~\micro l of cells.
Return the freezer stock to the -80\dc\ freezer, and place the O/N culture on a
shaker in the 37\dc\ room. This culture will be ready tomorrow!

It is a good practice to leave the uninoculated large culture on a shelf in the
37~\dc\ room over night as well so that in the morning it will be at the
correct temperature for growth.


\subsection{Batch growth -- the large culture}

On the day following the growth of the O/N culture, retreive it (and the
uninoculated large culture) from the 37\dc\ room. Typically bacterial cultures
are inoculated to a final optical density at 600~nm (\od) of 0.05-0.1. To be
able to do this one has to determine the \od\ of the O/N culture and inoculate
the larger culture with enough to achieve the correct OD. Note that any steps
in which cells are either removed or added to a culture are to be performed
under sterile conditions. \od\ determinations themselves do not need to be
sterile as the contents of the cuvettes are discarded afterwards.

After a night of growth the 50~ml culture should be at an \od\ of about 1.2. As
the linear range of most spectrophotometers is 0.1-1.0, the O/N culture will
need to be diluted to measure the \od\ (a 1:2 dilution should be sufficient).
Once the \od\ has been determined, inoculate the large culture to a final \od\
of 0.1, and place it on a shaker in the 37~\dc\ room. It may be useful to
double check that the culture is in fact at an \od\ of 0.1 after inoculation.

Bacterial cultures undergo four phases of growth in a batch culture:\\
\begin{enumerate} \item Lag phase -- this is a phase in which no growth takes
place and are traditionally thought to be adjusting the new medium\footnote{It
has been shown that in fact a small persister population is growing
exponentially in the background during this phase but the growth is masked by
the presence of a large static population which results in what appears to be a
``lag'' in growth; eventually the persister population grows large enough to
take over the culture.} \item Log phase -- at this stage the cells are growing
exponentially \item Stationary phase -- exhaustion of nutrients and the
accumulation of toxic metabolic products causes the cells to stop growing in
this phase \item Death phase -- continued lack of nutrients and exposure to
toxic compounds causes the cells to gradually die off \end{enumerate}

For kinetic assays cells are typically harvested in mid-log phase to ensure
that all the cells are configured similarly and are exhibiting maximal growth
without being subject to external effects from nutrient limitation or the
accumulation of toxins.

To determine when the cells have hit mid-log phase an \od\ determination (using
1~ml of culture, retreived under sterile conditions) should be made every
30~min. Plotting \od\ on a semilog axis will result in a straight line in
log-phase. \ec\ growing aerobically usually reach mid-log phase after about
3-4~hrs and for the purposes of this introduction can be harvested at an \od\
of 0.8.

\subsection{Cell harvesting} Cells will be harvested for subsequent experiments
by centrifugation. Sterile technique is no longer necessary from this stage
onward.

Retreive the 500~ml culture from the 37\dc\ room and decant it into a Beckman
500~ml centrifuge bottle. Weigh the complete bottle including the plug and cap
on a laboratory scale, and prepare a second bottle for balancing the centrifuge
rotor by filling it with water to the same weight (centrifuges typically have a
$\pm$1~g tolerance).

Centrifuge the two bottles using a JA-10 rotor at 7000~rpm for 10~min at 4\dc.
After centrifugation a large cell pellet will have formed on the bottom of the
bottle with cells. Pour off the supernatant, add about 20~ml of 100~mM TRIS buffer and
resuspend the pellet by holding the centrifuge bottle on a vortexer. After the
pellet is resuspended, transfer the cell suspension to a 50~ml falcon tube and
place it on ice. Divide the cell suspension into about 20 $\times$ 1~ml
aliquots in Eppendorf tubes.

Centrifuge the 20 Eppendorf tubes in a benchtop centrifuge at 13,500~rpm for 7
min, pour off the supernatants and store the tubes with cell pellets at -80\dc.
At this temperature cell pellets are stable for years.

\section{Kinetic assay} As this is an introduction to kinetic assays a simple
lactate dehydrogenase (LDH) assay will suffice. LDH converts pyruvate to
lactate whilst reoxidising NADH to NAD$^+$. Assays involving the light
spectrometer usually measure the consumpution or production of NADH. This often
means that a series of enzymes must be included in an assay to couple the
reaction of interest to NADH. This is not necessary with LDH as it directly
consumes NADH.

\subsection{Preparing a cell lysate}

Many methods are available for cell lysis with viable proteins involving glass
beads, sonication, french-pressing, bead-milling, and lysozyme. Grinding the
cells with glass beads is gentle enough to retain the activity of the enzymes
whilst yielding a large amount of protein.

To perform a glass bead extraction, cells are resuspended in a solution with
tiny glass beads and vortexed. Retrieve a cell pellet from the -80\dc\ freezer
and resuspend it in 1~ml of 100~mM TRIS buffer. The ideal ratio of glass beads
is 1~g/ml of cell suspension, so add 1~g of 0.1~\micro m glass beads to the
cell suspension, and transfer the whole suspension to a 1.5~cm diameter glass
test tube.

Perform the extraction by vortexing the test tube on full power for 6 minutes
in total whilst resting once every minute by placing the tube on ice for 15~s
to prevent heating of the sample. Note that typically it is advised to include
protease inhibitors during this stage to prevent the proteases released from
the cell's periplasm from degrading the enzymes of interest (e.g. PMSF). For
our purposes this is unnecessary as LDH is a relatively stable enzyme and shows
little degradation.

Once the grinding process is complete, pour the slurry into an Eppendorf tube
and centrifuge it using the benchtop centrifuge at 13,500~rpm for 5~min
(remember to balance it). Transfer the supernatant to a new tube and keep it on
ice; discard the tube with the glass beads.

\subsection{Lactate dehydrogenase assay}

pyruvate + NADH $\rightarrow$ lactate + NAD$^+$

The LDH assay will be performed by varying the concentration of the substrate
pyruvate until the enzyme is saturated with substrate, and measuring the rate
of the reaction at each of these steps.

Prepare the following solutions in 100~mM TRIS:

\begin{itemize}
\item 1~ml 200~mM pyruvate
\item 1~ml 4~mM NADH 
\end{itemize}

This assay will be performed using a 5-fold dilution series of the substrate
pyruvate in a 96-well plate with each of the five reaction wells having a final
volume of 100~\micro l. Prepare the following mix which excludes pyruvate and
is sufficient for ten reactions. Then pipette 90~ul into each of five wells on
a 96-well plate:

\begin{center}
\begin{tabular}[h]{rl} \toprule
10~\micro l & cell lysate\\ 
10~\micro l & 4~mM NADH (final conc. = 0.4~mM)\\ 
70~\micro l & 100~mM TRIS\\ \bottomrule
\end{tabular}
\end{center}

Now, prepare a 1:2 dilution series of pyruvate. This can be done in Eppendorf
tubes or conveniently in a series of adjacent wells on the 96-well plate by
diluting each step with 50~\% TRIS (bear in mind that the maximum volume of the
wells is 250~\micro l). The final concentrations in each tube/well should be
200, 100, 50, 25, 12.5~mM.

After inserting the plate and initialising the plate reader it is always good
practice to monitor the levels of NADH at 340~nm to ensure that they are
stable, and to allow the wells to equilibrate with the set temperature (25~\dc\
is fine). The other benefit of making the substrate dilution series in adjacent
wells is that the temperature is the same in the substrate and reaction wells.
The reaction can be initiated by pipetting 10~\micro l of pyruvate into each
reaction well. If a the substrate is in a series of wells then a multipipette
may be used, otherwise pipette each well individually starting at the lowest
concentration (to avoid losing too much data with the faster high concentration
reactions). Start the experiment which should measure NADH at 340~nm every 6~s
for about 10~min. Note that these assays are typically performed in triplicate
to be able to report a standard error.



\subsection{Protein determination -- the Bradford assay}

One final step is necessary: to determine the amount of protein present in the
cell lysate. This will be done using the Bradford assay. The Bradford reagent
contains Coomassie Brilliant Blue G-250 dye, which binds to protein and forms a
blue colour that can be measured at 595~nm. The following solutions will be
necessary for the assay:

\begin{itemize}
\item 1~ml 1~mg/ml bovine serum albumin (BSA)
\item a bottle of Bradford reagent stored in the laboratory refrigerator 
\end{itemize}

To perform the assay, a 4-fold 1:2 dilution series of the BSA must be made
either in Eppendorf tubes or in adjacent well of a 96-well plate. This will be
the protein standard with final concentrations of 1, 0.5, 0.25, 0.125~mg/ml.
Likewise a 4-fold 1:4 dilution series of the cell lysate must be made in
Eppendorfs or in the wells of a 96-well plate. 295~\micro l of the Bradford
reagent is pipetted into eight wells (four for the BSA standard, four for the
lysate), to which 5~\micro l of each protein solution is added (four
concentrations of BSA, four concentrations of lysate). After allowing the
reaction a few minutes to complete, measure the absorbance on a plate-reader at
595~nm.

By fitting a straight line function to the BSA standard points, the
concentration of the original cell lysate can be extrapolated back from the
cell lysate dilution series. A lysate dilution should be selected for
extrapolation that produces a reading within the range of the calibration curve
produced by the BSA dilution series. Note that these assays are typically
performed in triplicate to be able to report a standard error.


\end{multicols} \end{document}











