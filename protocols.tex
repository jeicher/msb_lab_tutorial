%        File: durps.tex
%     Created: Thu Jun 26 10:00 PM 2014 S
% Last Change: Thu Jun 26 10:00 PM 2014 S
%
\documentclass[a4paper]{article}
%\documentclass{scrartcl}
%\usepackage[utf8]{inputenc}
%\usepackage[english]{babel}
\usepackage{textcomp}
\usepackage{titlesec}
\usepackage{multicol}
\usepackage[cm]{fullpage}
\usepackage[]{graphicx}
\usepackage{booktabs}
\usepackage{tabularx}
\setlength{\parindent}{0pt}
\usepackage{txfonts}
%\usepackage{fancyhdr}
%\pagestyle{fancy}
%\renewcommand{\headrulewidth}{0pt}
%\cfoot{\textcopyright 2014 Johann Eicher}

\newcommand{\dc}{~$^{\circ}$C}
\newcommand{\ec}{\textit{E. coli}}



%\titlespacing*{\section}
%{0pt}{1.5ex plus 0.5ex minus .2ex}{1ex plus .2ex}
%\titlespacing*{\subsection}
%{0pt}{1.5ex plus 0.5ex minus .2ex}{1ex plus .2ex}

\begin{document} \begin{center} \huge Introductory tutorial to bacterial growth and kinetic assays in the Molecular
Systems Biology Group\\ \normalsize \end{center} \begin{multicols}{2}

This tutorial serves as a practical introduction to the typical experimental
work performed in the Molucalar Systems Biology lab.

\section{Sterile technique} 

When working with microorganisms almost all work is done under sterile
conditions to prevent contamination, and solutions prepared under non-sterile
conditions are subsequently sterilised by autoclaving. Contaminating bacteria,
yeast and fungi live on almost everything and float around in the laboratory
air. When using sterile technique, the following guidelines apply:

\begin{enumerate}
\item Work in the laminar flow hood which provides positive pressure (make sure the sterilising UV light is turned off whilst working the hood, and on otherwise).
\item Always wear latex gloves OR regularly sterilise hands with a 70~\% ethanol solution.
\item Flame bottlenecks with a bunsen burner whenever they are opened or used, and flame inoculating loops before each inoculation. Hockey-sticks should be kept in 70~\% ethanol and flamed before use.
\item Only use autoclaved pipette tips.
\end{enumerate}


\section{Growing \textit{Escherichia coli}} 

\ec\ is typically grown in LB medium for bulk preparation of cell matter,
however for reproducible physiological and/or kinetic studies a defined medium
such as M9 minimal medium has to be used. The following recipes can be scaled
to the desired volume.

To demonstrate the typical growth process of \ec\ the following steps will be
taken:

\begin{enumerate}
\item prepare either LB or M9 medium
\item inoculate a 5~ml culture from a freezer stock of \ec\ and allow it to grow overnight (O/N)
\item use the O/N culture to inoculate a large culture and incubate on a shaker at 37\dc
\item monitor cell growth using a spectrophotometer
\end{enumerate}

\subsection{Lysogeny broth (LB) medium} 

This is a rich medium composed of tryptic digests of casein (peptides), yeast
extract (vitamins, trace elements), and NaCl. As the exact composition of
constituents in tryptone and yeast extracts is unknown and will vary per batch,
LB medium is not recommended for physiological and kinetic studies, but rather
for bulk growing of bacteria (e.g. for protein expression).

To prepare 1~l of LB medium, dissolve the following components in water
(distilled water or preferably milli-Q) in a beaker with a magnetic spinner:

\begin{center}
\begin{tabular}[h]{rc} \toprule
10~g    &tryptone \\ 
5~g     &yeast extract\\ 
10~g    &NaCl\\ \bottomrule
\end{tabular}
\end{center}

As a precaution it is possible at this stage to adjust the pH of the solution
but is probably unnecessary. The solution can now be decanted into an
Erlenmeyer flask with a cotton wool bung and foil covering the neck (note that
for adequate aeration a 1~l culture should be grown in a 3~l Erlenmeyer flask)
and autoclaved (usually at 121\dc for 15-20 min).

\subsection{M9 minimal medium}

M9 is a minimal medium with clearly defined components ideal for physiological
and kinetic studies. Minimal media force the organism to produce the majority
of the required metabolites, vitamins, co-factors endogenously and only
provides the bare minimum of components exogenously in the medium. Unlike LB
medium, M9 medium is buffered by the two phosphate components HPO$_4^{2-}$ and
H$_2$PO$_4^{1-}$. It is thus imperative that care is taken during preparation to
include the correct forms of the phosphate salt!

To prepare 1~l of M9 medium, dissolve the following in 977~ml of water (milli-Q is
preferred for reproducibility) in a beaker with a magnetic spinner:

\begin{center}
\begin{tabular}[h]{rc} \toprule
12.8 g  & Na$_2$HPO$_4$.7H$_2$O \textbf{OR}\\
6.0 g  & Na$_2$HPO$_4$\\\midrule
3.0 g  & KH$_2$PO$_4$\\
0.5 g   & NaCl\\
1.0 g   & NH$_4$Cl\\ \bottomrule
\end{tabular}
\end{center}

Also prepare a 20\% w/v glucose solution in milli-Q water (typically glucose,
sodium gluconate or glycerol are used as carbon sources); 500~ml is enough for
25~l of M9 medium.  Glucose solutions may take a while to dissolve fully, but
if left stirring with a magnetic stirrer the clumps will eventually disperse.

Transfer the glucose solution to a Schott bottle and the M9 salts to an
Erlenmeyer flask with a cotton-wool bung and foil covering the neck of the
flask (note that for adequate aeration a 1~l culture should be grown in a 3~l
Erlenmeyer flask). Autoclave these solutions (usually at 121\dc for 15-20 min).
The reason for autoclaving the salts and sugars separately is that when
autoclaved together the sugars are typically caramelised, especially when
phosphates are present.

While the glucose and salt solutions are being autoclaved, prepare the
following two constituents. Once prepared, they should be filter sterilised
using a 50~ml syringe and a 0.2~$\mathrm{\muup m}$ syringe filter into 50~ml
Falcon tubes and stored at 4\dc\ (use a new filter for each solution):

\begin{itemize}
\item 50~ml 1~M MgSO$_4$
\item 50~ml 0.1~M CaCl$_2$
\end{itemize}

To prepare 1~l of M9 medium add the following under sterile conditions to the
977~ml M9 salts autoclaved salts in the Erlenmeyer flask:

\begin{center}
\begin{tabular}[h]{rc} \toprule
20~ml   & 20~\% carbon source (0.2~\% final)\\
2~ml    & 1~M MgSO$_4$\\
1~ml    & 0.1~M CaCl$_2$\\ \bottomrule
\end{tabular}
\end{center}

As a precaution it is possible at this stage to adjust the pH of the solution
which will be buffered around 7.2. This is generally unncecessary for growth of
\ec.


\subsection{Batch growth}
Typical aerobic bacterial batch cultures are grown by shaking in Erlenmeyer flasks. 




\end{multicols} \end{document}


